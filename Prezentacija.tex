\documentclass{beamer}
\usepackage{beamerthemeshadow}
\usepackage{graphicx}
\usepackage{color}
\usepackage[utf8]{inputenc}
\usepackage{hyperref}
\usepackage[flushleft]{threeparttable}
\usepackage{csquotes}
\usepackage[english,serbian]{babel}
\usetheme{Hannover}
\usecolortheme{wolverine}
% \setbeamertemplate{headline}{}
\setbeamersize{sidebar width left=0pt}
\setbeamertemplate{sidebar left}{}
% \definecolor{beamer@green}{rgb}{0.5, 2, 0.5}
% \setbeamercolor{structure}{fg=beamer@green}

\def\dJ{{\fontencoding{T1}\selectfont\dj}}
\def\Dj{{\fontencoding{T1}\selectfont\DJ}}

\begin{document}
\title{Iterativna lokalna pretraga}
\author[]{Aleksa Voštić, Lazar Perišić,\\ Anđela Križan, Anđela Janošević}
\institute{Matematički fakultet\\Univerzitet u Beogradu}
\date{
	\footnotesize{Beograd, 2020.}	
}

\begin{frame}
	\thispagestyle{empty}
	\titlepage
\end{frame}

%-----------------------------------------------------------------

\section*{Uvod}
\begin{frame}[fragile]
	\frametitle{Uvod}
	\begin{itemize}
		\item Metaheurističke metode za rešavanje teških optimizacionih problema
		\item Prilikom dizajniranja metaheuristike, poželjno je da bude jednostavna, efikasna, opšte namene
		\item Idealan slučaj je kada se metaheuristika može koristiti bez ikakvog znanja o zavisnosti od problema
		\item Znanje specifično za problem mora biti inkorporirano u metaheuristiku da bi se dostiglo vrhunsko stanje
		\item Pokušavamo da dekomponujemo metaheuristički algoritam na nekoliko delova:
		\begin{itemize}
			\item potpuno opšti namenski deo
			\item svako znanje specifično za problem ugrađeno u metaheuristiku bilo bi odvojeno u drugi deo
		\end{itemize}
	\end{itemize}

\end{frame}

\section*{Uvod}
\begin{frame}[fragile]
	\frametitle{Uvod}
	\begin{itemize}
		\item Iterativna lokalna pretraga pruža jednostavan način da se zadovolje svi ovi zahtevi
		\item Suština iterativne lokalne pretrage je da se izbegne zaglavljivanje u lokalnom minimumu tako što u više iteracija primenjuje lokalnu pretragu na novo generisano početno rešenje
		\item Ova ideja ima dugu istoriju, a njeno ponovno otkriće od strane mnogih autora dovelo je do mnogo različitih imena za iterativnu lokalnu pretragu poput iterativnog spusta, Markovljevi lanci velikog koraka, iterativni Lin-Kernigan, lančana lokalna optimizacija...
	\end{itemize}

\end{frame}
%------------------------------------------------------------------------

\section{Ideja iza iterativne lokalne pretrage}
\begin{frame}
	\frametitle{Ideja iza iterativne lokalne pretrage} 

\end{frame}
%-----------------------------------------------------------------------

\section{Implementacija iterativne lokalne pretrage}
\begin{frame}[fragile]\frametitle{Implementacija iterativne lokalne pretrage}
 
\end{frame}

% \subsection{Početno rešenje}
\begin{frame}[fragile]\frametitle{Početno rešenje i Perturbacija}
	\subsection{Početno rešenje}
	\subsection{Perturbacija}

\end{frame}

% \subsection{Perturbacija}
% \begin{frame}[fragile]\frametitle{Perturbacija}

% \end{frame}


% \subsection{Kriterijum prihvatanja}
\begin{frame}[fragile]\frametitle{Kriterijum prihvatanja i Lokalna pretraga}
	\subsection{Kriterijum prihvatanja}
	\subsection{Lokalna pretraga}
\end{frame}

% \subsection{Lokalna pretraga}
% \begin{frame}[fragile]\frametitle{Lokalna pretraga}

% \end{frame}

\section{Primene iterativne lokalne pretrage}
\begin{frame}[fragile]\frametitle{Primene iterativne lokalne pretrage}

\end{frame}

\subsection{Problem trgovačkog putnika}
\begin{frame}[fragile]\frametitle{Problem trgovačkog putnika}

\end{frame}

\subsection{Problemi raspoređivanja}
\begin{frame}[fragile]\frametitle{Problemi raspoređivanja}
	\subsubsection{Single machine total weighted tardiness problem}
	\subsubsection{Flow shop problem}
	\subsubsection{Job shop scheduling problem}
\end{frame}

\section{Efektivnost i efikasnost ILS algoritma}
\begin{frame}[fragile]\frametitle{Efektivnost i efikasnost ILS algoritma}

\end{frame}

\section{Zaključak}
\begin{frame}[fragile]\frametitle{Zaključak}
	\begin{itemize}
		\item ILS poseduje mnoge poželjne karakteristike metaheuristike: jednostavan je, lagan za implementaciju, robustan i veoma efikasan
		\item Suštinska ideja ILS-a leži u fokusiranju pretraživanja ne na celokupnom prostoru rešenja, već na manjem potprostoru koji je definisan rešenjima koja su lokalno optimalna za datu optimizaciju
		\item  Koliko će se ovaj pristup pokazati efikasnim, uglavnom zavisi od izbora lokalne pretrage, perturbacija i kriterijuma prihvatanja
		\item Zbog svojih karakteristika verujemo da je ILS obećavajući i moćan algoritam za rešavanje stvarnih kompleksnih problema
	\end{itemize}

\end{frame}

\begin{frame}[fragile]\frametitle{Literatura}
	\thispagestyle{empty}

\end{frame}


\end{document}
